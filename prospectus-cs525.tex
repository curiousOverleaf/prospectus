\documentclass{proc}

\begin{document}

%\title{Perception of likelihood in social visualization}
%\title{cross language analysis of Data viz interaction}
%\tilde{analysis of social discourse used in risk perception}
\title{Perception and Language: Analysis of popular discourses in uncertainty visualization}

%\title{color-based perception of risk}
%We vary the color, maps projection, scale, and see how people classify the situation as risky...

\author{Author1, Author2}

\maketitle

%----------------------------------------------------------SECTION---------
\section{Introduction}

We encounter different forms of visualizations in many aspects of our daily lives and there are different factors that influence how we interact with or make use of them: information about designers or data origin can influence the trust that viewers have in a visualization \cite{peck2019data}, social information such as popular choices can influence our visual judgments\cite{hullman2011impact}, literacy, familiarity with charts or with the interface design \cite{blascheck2018exploration} can influence the way we interact with a visualization system and extract information from it, the diversity of languages and terminologies for visual components such as colors \cite{lindsey2009world} could explain how visualizations are build across cultures and countries. 

Visualization can be a powerful tool for empowering people and supporting them in their decision-making. However knowledge and information contained in data and encoded in the visuals need to be communicated in order to make them actionable. Language is an important tool used to formulate and communicate the content of a visualization. When working with complex/irregular/? visualizations such as those that embed uncertainty, or show critical information, differences in terminologies among groups, across cultures and geographical location, can make the message ambiguous. 
 
In this experiment, we want to study the discourse that people structure and use to communicate information such as risk, uncertainty or gravity (seriousness) in a (social?) data visualization. 

What is a social data visualization? 
\section{One-sentence description}

Analyzing people's behaviors towards specific visualizations help us understand how language and discourse shape the understanding or communication of findings in visualization -- how people understand concepts in vis and which discourse they use to communicate them. 

%----------------------------------------------------------SECTION---------
\section{Project Type}
Crowdsourcing, large experiment.

%----------------------------------------------------------SECTION---------
\section{Audience} 
\begin{quote}
\textit{Who is the audience for this project? 
How does it meet their needs? 
What happens if their needs remain unmet?}
\end{quote}
The audience of the projects are people who encounter data visualizations in their daily live, use them to make decisions, but especially to communicate findings. It is necessary to know the popular discourse used to communicate or describe critical visualization. Understanding the language one uses or others use for describing graphs and phenomena such as uncertainty, risk or probability can help people make informed personal or collective decisions. (putting words on graphs or graphs on words)

%----------------------------------------------------------SECTION---------
\section{Approach}
The experiment will be conducted using online crowdsourcing. 
It will have two components. One component will investigate how people understand or represent data visualization concepts in daily discourse? another one will investigate which discourse do people use to communicate information in visualization. 

\subsection{Details}
\begin{quote}
\textit{What is your approach?}
Trial + real experiments. Trial objectives: train subjects on how to construct visualization.
\end{quote}
The experiment could have two phases: 

\subsubsection*{visualize uncertainty, gravity, risk, mean, range etc}
(maybe stick with only one of these )
Participants will be given interactive visualizations as well as texts or verbal description of situations that carry information such uncertainty, probability, confidence, gravity of risk or phenomenon, mean, variance (to be decided). Participants will have to adjust the visualization (say, sliding ) to the what they think would represent the idea in the closest way. 

\subsubsection*{Describe specific visualization}
Participants will be ask to describe the uncertainty or risks (...) that they see frmo a visualization. Participants could be given a list of words to choose among or they can be free to come up with their own terminologies. 


\subsection{Evidence for Success}
\begin{quote}
\textit{Why do you think it will work?} 

\end{quote}
% the impact of the combination discourse \& visualization on decision-making 

%----------------------------------------------------------SECTION---------
\section{Best-case Impact Statement}
\begin{quote}
\textit{In the best-case scenario, what would be the impact statement (conclusion statement) for this project?}
\end{quote}
Ideally: "We found out that when it comes to uncertainty/risk/ visualization, these are the most common language/discourse that people associate to this different values of mean/range/criticality"

\section{Major Milestones}
Beside selecting  the crowdsourcing platform to use and learning more about crowdsourcing techniques or design, mostly via readings and literature search: 
\begin{itemize}
	\item select the topic in visualization to do the study on, narrow down the topic
	\item implement the visualizations, prepare the texts for the tasks. 
	\item deploy experiment(run tests, run main experiments)
	\item iterate
	\item analyze data and draw conclusion
\end{itemize}

%----------------------------------------------------------SECTION---------
\section{Obstacles}

\subsection{Major obstacles} % (if these fail, the project is over)
\begin{itemize}
    \item not sufficient participants to draw conclusion from
    \item 
\end{itemize}
\subsection{Minor obstacles}
\begin{itemize}
    \item  
\end{itemize}

%----------------------------------------------------------SECTION---------
\section{Resources Needed}
\begin{quote}
\textit{What additional resources do you need to complete this project?}
\end{quote}
\begin{itemize}
    \item Amazon Mechanical turk 
\end{itemize}

%----------------------------------------------------------SECTION---------
\section{Related Work}
\begin{quote}
\textit{List 5 major publications that are most relevant to this project, and how they are related (sample citation.}
\end{quote}

Papers that studied external influences on our behaviours with charts or how we build visualization:

Hullman et al.'s work~\cite{hullman2011impact} show that external influences can also modify our perception of visualizations.

PApers that studied internal influences on our behaviors towards data visualization
Lindsay et al.'s work~\cite{lindsey2009world} on color namings across different languages

Peck et al.'s work~\cite{peck2019data} analysed how individuals react to or question their trust in charts based on their personal experience or personal connection to the context of the data. 

Harrison et al.'s work~\cite{harrison2012exploring} on the impact of emotional priming on individual's performance on visualization tasks. 

Language is both internal and external as it is a tool that we use to extract information from charts and communicate them.
%----------------------------------------------------------SECTION---------
\section{Define Success}
\begin{quote}
\textit{What is the minimum amount of work necessary for this work be publishable?}
\end{quote}

\bibliographystyle{abbrv}
\bibliography{prospectus}
\end{document}
